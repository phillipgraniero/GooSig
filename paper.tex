\documentclass[11pt]{article}
\usepackage[hidelinks]{hyperref}
\usepackage{amsthm,amsmath,amsfonts,amssymb,amstext,mathtools}
\usepackage{fullpage}
\usepackage{macros}



\begin{document}

\title{An Air Drop That Preserves Recipient Privacy}
\author{Dan Boneh \and Riad Wahby}

\maketitle

\begin{abstract}
\end{abstract}

%%%%%%%%%%%%%%%%%%%%%%%%%%%%%%%%%%%%%%%%%%%%%%%%%%%
\section{Introduction}

More to come.


%%%%%%%%%%%%%%%%%%%%%%%%%%%%%%%%%%%%%%%%%%%%%%%%%%%
\section{Private air drop: definitions}


A private air drop (PAD) system is a triple of efficient algorithms:
\begin{itemize}
\item $\text{sendTokens}(\PK) \to (c, s)$:
Generates a token $c$ to be posted on the blockchain,
and a secret $s$ given to the recipient via a private channel
(e.g., by posting an encryption of $s$ encrypted under $\PK$ 
to the blockchain). 

\item $\text{sign}(\SK, (c, s), m) \to \text{sig}$:
Signs a message $m$ indicating that the token $c$ is spent.

\item $\text{verify}(c, m, \text{sig}) \to \{\text{yes},\text{no}\}:$
Verifier that a signature $\text{sig}$ on token $c$ is valid.
\end{itemize}

Informal security properties: 
(1) $c$ and $\text{sig}$ should reveal nothing about $\PK$,
(2) without $\SK$: existential unforgeability under a 
(one-time) chosen message attack.



%%%%%%%%%%%%%%%%%%%%%%%%%%%%%%%%%%%%%%%%%%%%%%%%%%%
\section{A private air drop to RSA public keys}

From here we use the following notations:
$\lambda$ is the security parameter (e.g., $\lambda = 128$), 
$\range{\ell}$ denotes the set of integers $\{0,1,\ldots,\ell-1\}$ and 
$\primes$ is a set of size $2^\lambda$ 
containing the smallest $2^\lambda$ odd primes. 
%
We will also need a set of public parameters that include the description
of a cyclic group $\GG$ of unknown order.  We let $g$ and $h$ be two
random generators of $\GG$ where the discrete-log of $g$ base $h$ is unknown. 

\subsection{Proving knowledge of the factorization of a hidden integer}

We first develop a zero knowledge protocol for proving
knowledge of the factorization of an integer $n$ given
as a commitment to $n$.
We let $N$ be a known upper bound on $n$, so that $n \in \range{N}$.
For simplicity we will also assume that $\abs{\GG} < N$. 
The given commitment to $n$ is $c \deq g^n \cdot h^s \in \GG$,
where $s \rgets \range{N}$.

To prove knowledge of the factorization of $n$ the prover
proves knowledge of a $w \in \range{N}$ such that $w^2 \equiv t \pmod{n}$
for some small public prime $t \in \ZZ$, namely $2 \leq t \leq \lambda$.
That is, we provide a ZKPK for the relation
\begin{equation} \label{eq:lang}
  {\cal R} \deq \left\{ \Bigl( (c, t) \in \GG \times \range{\lambda},\ \ 
                              (n, s, w, a) \in \range{N}^4 \Bigr)  \ \ 
        \text{s.t.}\ \ 
                \begin{split}
                        c   & = g^n \cdot h^s \ \ \text{in $\GG$}, \\
                        w^2 & = t + a \cdot n \ \ \text{in $\ZZ$}, \\
                        t   & \ \textrm{a prime}
                \end{split}  \right\}.
\end{equation}
Here $(c, t)$ is the statement and $(n, s, w, a)$ is the witness.
The integer relation $w^2 = t + a n$ proves that 
$w^2 \equiv t \pmod{n}$, as required.
We note that a ZKPK for~\eqref{eq:lang} proves that $t \in \ZZ$ is a quadratic
residue modulo the committed $n$.  Leaking this bit of information about $n$
does not interfere with the application to a secure air drop.

\medskip\noindent
The ZKPK protocol for~\eqref{eq:lang} 
between the prover $P$ and the verifier $V$ works as follows:
\begin{itemize}
\item input:  the verifier has $(c, t) \in \GG \times \range{\lambda}$ and 
the prover has $(c, t, n, s, w, a) \in \GG \times \range{N}^5$  

\item Prover $P$ chooses a random integer $s_1 \rgets \range{N}$ and computes
       $c_1 \gets g^w \cdot h^{s_1} \in \GG$. 

\item Next, define a homomorphism $\phi:\ZZ^7 \to \GG^3 \times \ZZ$ as follows:
\begin{equation} \label{eq:hom}
  \phi(w, \mathit{w2}, s_1, a, \mathit{an}, \mathit{s1w}, \mathit{sa}) \deq
      \Bigl(\ 
       g^w \cdot h^{\mathit{s1}},\ \ \ 
       g^{\mathit{w2}} \cdot h^{\mathit{s1w}} / c_1^w,\ \ \ 
       g^{\mathit{an}} \cdot h^{\mathit{sa}} / c^a,\ \ \ 
       \mathit{w2} - \mathit{an}\ \Bigr)
\end{equation}
It is easy to see that $\phi$ is a group homomorphism.
The range of $\phi$ is the group $\GG^3 \times \ZZ$.  
We will write the group operation in this group multiplicatively.
That is, if $(a_i, b_i, c_i, d_i) \in \GG^3 \times \ZZ$ for $i=1,2$
then 
\[  (a_1, b_1, c_1, d_1) \cdot (a_2, b_2, c_2, d_2) \deq 
      (a_1 a_2,\ b_1 b_2,\ c_1 c_2,\ d_1 + d_2).  \]

We need a ZKPK for a $\phi$-preimage of 
$T \deq (c_1, 1, 1, t) \in \GG^3 \times \ZZ$. 
In other words, we need a zero-knowledge proof of knowledge for a vector 
$\vv' = (w', \mathit{w2}', s_1', a', \mathit{an}', 
              \mathit{s1w}', \mathit{sa}') \in \ZZ^7$ 
such that 
\[   \phi(\vv') = T = (c_1, 1, 1, t) \in \GG^3 \times \ZZ.  \]
Doing so proves 
that $c_1$ is a commitment to $w' \in \ZZ$, 
that $\mathit{w2}' = (w')^2$, and 
that $\mathit{an}' = a' \cdot n$ for some integer $a'$.
Then the last relation proves that $(w')^2 - a' \cdot n = t$ in $\ZZ$,
as required. 

We design a ZKPK for a $\phi$-preimage 
using a zero-knowledge protocol from~\cite[\S 3.5]{ourpaper}:
\begin{enumerate}
\item The prover chooses a random $\rv \rgets \range{N}^7$
and computes $\Rv \gets \phi(\rv) \in \GG^3 \times \ZZ$. \\
It sends $(c_1, \Rv)$ to the verifier.

\item The verifier chooses challenges $\mathit{ch} \rgets \range{2^\lambda}$
and $\ell \rgets \primes$. \\
It sends $(\mathit{ch}, \ell)$ to the prover.

\item The prover computes:
\[  \xv \gets (\mathit{ch} \cdot \vv + \rv) \in \ZZ^7, \quad
    \zv_\ell \gets (\xv \bmod \ell) \in \range{\ell}^7, \quad
    \zv_q \gets \lfloor \xv / \ell \rfloor \in \ZZ^7, \quad
    \Zv_q \gets \phi(\zv_q)  
\]
and sends $(\Zv_q, \zv_\ell) \in \GG^3 \times \ZZ \times \range{\ell}^7$ 
to the verifier. 

\item The verifier accepts if 
$\ \Zv_q^\ell \cdot \phi(\zv_\ell) = T^\mathit{ch} \cdot \Rv\ $ in 
$\GG^3 \times \ZZ$.
\end{enumerate}
\end{itemize}

\paragraph{Comment.}
There is a seemingly simpler method to prove that one knows the
factorization of~$n$, namely, prove that one knows $p$ such that $p$
divides $n$.  That is, instead of committing to $w$ we would commit to
$p$ and then prove that the committed value divides $n$. But we would
also need to prove that the committed factor is not one of 
$\{1, -1, n, -n\}$, and that results in a longer proof.

\paragraph{Proof of security.}
We need to prove the protocol above is zero-knowledge and a 
proof of knowledge. 

\paragraph{Proof of zero-knowledge.}


\paragraph{Proof of extractability.}



\subsection{The concrete air drop system}

The air drop system is obtained by applying the Fiat-Shamir
heuristic to the protocol from the previous section. 
The concrete private air drop system works as follows:

\paragraph{Algorithm $\text{sendTokens}(\PK)$:}
To send tokens to an RSA public key $\PK = (n, e)$ do:
\begin{itemize}
\item 
generate a random $s \rgets \range{G}$.

\item 
output $(c, s)$ where $c \gets g^n \cdot h^s \in \GG$.
\end{itemize}

\medskip\noindent 
The following two notes can help optimize the scheme:
\begin{itemize}
\item[$-$] $s$ is privately sent to the owner of $\PK$, for example
by encrypting $s$ under $\PK$ to obtain a ciphertext $c'$ and
publishing this $c'$ somewhere, not necessarily on the blockchain.  There is
a public association between $c'$ and $\PK$ (e.g., via a hash table),
so that the owner of $\PK$ can easily find $c'$.  One can include
$H(c)$ in the plaintext encrypted in $c'$ to help the owner of $\PK$
quickly find $c$.  

\item[$-$] One can generate $s$ 
as $s \gets H(s')$ where $s'$ is a random 256-bit value and $H$ is a hash
function $H:\{0,1\}^{256} \to \range{G}$.  The ciphertext $c'$ is
an encryption of $s'$, instead of $s$, which can shrink~$c'$.
\end{itemize}


\paragraph{Algorithm $\text{sign}(\SK, (c, s), m)$:}  
To withdraw the funds associated with the token $c$ the user
signs a withdrawl message $m$ using the RSA secret key $\SK = (n,p,q)$ 
as follows:
\begin{itemize}
\item find a prime $2 \leq t \leq \lambda$  such that $t$ is a quadratic 
residue in $\ZZ_n$. 

\item find integers $(w,a)$ such that $w^2 = t + a n$ in $\ZZ$
           (i.e. $w$ is a square root of $t$ modulo $n$)

\item choose a random $s_1 \rgets \range{N}$ and compute
           $c_1 \gets g^w \cdot h^{s_1} \in \GG$.

\item choose a random $\rv \rgets \range{N}^7$
and compute $\Rv \gets \phi(\rv) \in \GG^3 \times \ZZ$.

\item compute $\mathit{seed} \gets \text{Hash}(m, \GG, g, h, c, c_1, t, \Rv$).

\item
use a $\text{PRNG}(\mathit{seed})$ to generate $(\mathit{ch}, \ell)$, 
where $\mathit{ch} \in \range{2^\lambda}$ and $\ell \in \primes$.

\item compute:
\[  \xv \gets (\mathit{ch} \cdot \vv + \rv) \in \ZZ^7, \quad
    \zv_\ell \gets (\xv \bmod \ell) \in \range{\ell}^7, \quad
    \zv_q \gets \lfloor \xv / \ell \rfloor \in \ZZ^7, \quad
    \Zv_q \gets \phi(\zv_q) \in \GG^3 \times \ZZ.
\]

\item output the signature 
$\mathit{sig} = (c_1, t, \mathit{ch}, \ell, \Zv_q, \zv_\ell)$.
\end{itemize}



\paragraph{Algorithm $\text{verify}(c, m, \mathit{sig}):$}
Verify that $\mathit{sig} = (c_1, t, \mathit{ch}, \ell, \Zv_q, \zv_\ell)$ 
is a valid signature for token~$c$ on the message~$m$.
\begin{itemize}
\item reject if $t \notin \range{\lambda}$ or $t$ is not a prime.

\item with $T \deq (c_1, 1, 1, t) \in \GG^3 \times \ZZ$, 
compute $\Rv' \gets \Zv_q^\ell \cdot \phi(\zv_\ell) / T^\mathit{ch} \in \GG^3 \times \ZZ$.

\item compute $\mathit{seed}' \gets \text{Hash}(m, \GG, g, h, c, c_1, t, \Rv'$)

\item use a $\text{PRNG}(\mathit{seed}')$ to generate 
  $\mathit{ch}' \in \range{2^\lambda}$ and $\ell' \in \primes$. \\
Since $\ell$ is
  included in the signature, the verifier does not need to search for
  a prime, but can instead verify that the given $\ell$ is prime and
  is not too far from the base starting point of the prime search.

\item accept iff $\mathit{ch}' = \mathit{ch}$ and $\ell' = \ell$.
\end{itemize}



\paragraph{Comments.}
This scheme is only secure is a group where there are no known
elements of low order.  The group $\ZZ/n$ is not such a group because
of the element $-1$ of order $2$.  To eliminate this element we work
in the quotient group $(\ZZ/n)/\{\pm 1\}$, representing elements as
$\abs{x} = \min(x, n - x)$.  In this group $-1$ is the same as $1$,
and presumably there are no other known elements of known order.


\paragraph{Proof of security.}
We need to prove two things: 
(1) privacy: there is a simulator that can generate a properly distributed
signature just given $c$ and $m$, and 
(2) security: an attacker that can generate a valid siganture for $c$ using
some message $m' \neq m$ can be used to compute $\sqrt{t} \in \ZZ_n$
for some $2 \leq t \leq 1000$. 

\paragraph{Proof of privacy.}


\paragraph{Proof of security.}
 


%%%%%%%%%%%%%%%%%%%%%%%%%%%%%%%%%%%%%%%%%%%%%%%%%%%
\section{A private air drop to ElGamal public keys}



\bibliographystyle{alpha}
\bibliography{paper}

\end{document}
