\section{Definitions}\label{sec:defs}

\subsection{Private airdrop scheme}

\paragraph{High-level description} In a private airdrop, a \emph{sender} \Sr creates a \emph{token}
    for a \emph{recipient} \Rt whose public key is $\PK$, and posts
    the token in a transaction on a blockchain.
To claim the transaction's value, \Rt generates a new transaction and
    uses the token and corresponding secret key $\SK$ to generate a signature; it posts the
    new transaction and the signature to the blockchain.
Any \emph{verifier} \Vr can check the signature on the transaction
    using the token, \emph{without} learning $\PK$.

\paragraph{Syntax} A private airdrop scheme PAD is a triple of efficient algorithms:
%
\begin{itemize}
\item $\text{sendTokens}(\PK) \to (c, s)$:
%
\Sr generates token $c$ and secret $s$, then posts $c$ publicly and sends $s$
    to \Rt via some private channel.
%To send a \emph{token} $c$ to a \emph{recipient} \Rt whose key is $(\PK,
%    \SK)$, a \emph{sender} \Sr generates $c$ to be posted on a blockchain,
%    plus a \emph{secret}~$s$ given to \Rt via some private channel.
\unskip\footnote{For example, a encryption of $s$ under $\PK$ can be posted
    to a public messageboard. Note, however, that the posted ciphertext
    must not publicly associate $\PK$ with the token $c$.}
%\RSW{is it necessary to suggest that Enc(s) appear on the blockchain? It could just be posted publicly, since it doesn't need to be remembered for long.}

\item $\text{sign}(\SK, (c, s), m) \to \sig$:
\Rt generates signature $\sig$ over message $m$.

\item $\text{verify}(c, m, \sig) \to \{\text{yes},\text{no}\}:$
\Vr checks signature $\sig$ on message $m$ for token $c$.
\end{itemize}

\paragraph{Security} A private airdrop scheme is secure if it is \emph{anonymous} and \emph{unforgeable}.

\medskip\noindent\textit{Anonymity} means, roughly speaking, that $c$ and $\sig$ reveal
    nothing about $\PK$.
Specifically, consider $\mathsf{Anonym}$, a game between a PPT adversary $\adv$
    and a challenger $\chall$:
%
\begin{itemize}
\item $\chall$ chooses keys $(\PK_0, \SK_0)$, $(\PK_1, \SK_1)$, and sends $(\PK_0, \PK_1)$ to $\adv$.

\item $\adv$ chooses $\msg$ and sends it to $\chall$.

\item $\chall$ chooses $b \rgets \{0,1\}$,
computes $(c, s) \gets \text{sendTokens}(\PK_b)$ and
$\sig \gets \text{sign}(\SK_b, (c, s), \msg)$,
and sends $(c, \sig)$ to $\adv$.

\item $\adv$ outputs $b'$.

\end{itemize}
%
$\adv$ wins $\mathsf{Anonym}$ if $b' = b$.

\begin{definition}
Let an adversary $\adv$'s advantage in $\mathsf{Anonym}$ be
$\mathrm{Adv}_{\adv}^{\mathsf{Anonym}} = \sizeof{\probab{\adv~\text{wins}} - \nicefrac{1}{2}}$.
Then an airdrop scheme is \emph{anonymous} if, for any PPT adversary
    $\adv$, $\mathrm{Adv}_{\adv}^{\mathsf{Anonym}} \le \text{negl}$.
\end{definition}


\medskip\noindent\textit{Unforgeability} means, roughly speaking, that
    a party that does not know $\SK$ cannot generate a valid signature
    for any message, even if it knows a valid signature for any other
    message.
Specifically, consider $\mathsf{Unforg}$, a game between a PPT adversary
    $\adv$ and a challenger $\chall$:
%
\begin{itemize}
\item $\chall$ chooses a key $(\PK, \SK)$ and sends $\PK$ to $\adv$.

\item $\adv$ computes $(c, s) \gets \text{sendTokens}(\PK)$, chooses \msg, and sends $(c, s, \msg)$ to $\chall$.

\item $\chall$ computes $\sig \gets \text{sign}(\SK, (c, s), \msg)$ and sends to $\adv$.

\item $\adv$ outputs $(\msg', \sig')$.

\end{itemize}
%
$\adv$ wins $\mathsf{Unforg}$ if $\msg' \neq \msg \wedge \text{verify}(c, \msg', \sig') = \text{yes}$.

\begin{definition}
Let an adversary $\adv$'s advantage in $\mathsf{Unforg}$ be
$\mathrm{Adv}_{\adv}^{\mathsf{Unforg}} = \probab{\adv~\text{wins}}$.
Then an airdop is \emph{unforgeable} if, for any PPT adversary $\adv$,
$\mathrm{Adv}_{\adv}^{\mathsf{Unforg}} \le \text{negl}$.
\end{definition}

