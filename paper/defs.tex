\section{Private airdrop: definitions}\label{sec:defs}

A private airdrop (PAD) system is a triple of efficient algorithms:
\begin{itemize}
\item $\text{sendTokens}(\PK) \to (c, s)$:
Generates a token $c$ to be posted on the blockchain,
and a secret~$s$ given to the recipient via a private channel.
For example, an encryption of $s$ under $\PK$ can be posted
to the blockchain.   The posted encrypted $s$ is not associated with 
the token $c$. 
\RSW{is it necessary to suggest that Enc(s) appear on the blockchain? It could just be posted publicly, since it doesn't need to be remembered for long.}

\item $\text{sign}(\SK, (c, s), m) \to \sig$:
Signs a message $m$ indicating that the token $c$ is spent.

\item $\text{verify}(c, m, \sig) \to \{\text{yes},\text{no}\}:$
Verify that a signature $\sig$ on token $c$ is valid.
\end{itemize}

TODO: Need to state security properties formally: 
(1) $c$ and $\text{sig}$ should reveal nothing about $\PK$,
(2) without $\SK$ we have existential unforgeability under a 
(one-time) chosen message attack.

\subsection{Zero-knowledge proofs in generic groups}

TODO: Need overview of zero-knowledge in generic groups. 

\paragraph{The representation extraction lemma.}
Let $\adv$ be an algorithm that outputs 
a generic group element $u \in \GG$.
The following lemma shows that there is an extractor that
can extract from $\adv$ an integer representation of $u$
relative to a supplied set of group generators.
Moreover, this integer representation is unique.

\begin{lemma}[Unique representation extraction in generic groups]
\label{lem:unique}
Let $\adv_1, \adv_2$ be two randomized algorithms that interact
with group oracles for a generic group $\GG$ of unknown order.
For $i=1,2$, algorithm $\adv_i$ outputs $u_i \in \GG$.
Suppose that each algorithm makes at most $q$ type-1 queries
and let $g_1,\ldots,g_q \in \GG$ be the returned random group elements. 

There is an extractor $\bdv$ that emulates 
the generic group oracles for $\adv_1$ and $\adv_2$
such that for $i=1,2$ when $\bdv$ interacts with $\adv_i$ 
the following holds with overwhelming probability:
if $\adv_i$ outputs $u_i \in \GG$ 
then the extractor $\bdv_i$ outputs a representation
$\alpha_{i,1},\ldots,\alpha_{i,q} \in \ZZ$
such that $u_i = g_1^{\alpha_{i,1}} \cdots g_q^{\alpha_{i,q}}$.
Moreover, if $u_1 = u_2$ then the two representations are the same,
namely $\alpha_{1,j} = \alpha_{2,j}$ for $j=1,\ldots,q$. 
\end{lemma}

