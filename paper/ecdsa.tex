\section{A private airdrop to ECDSA and Schnorr public keys}\label{sec:dsa}

\rsw{this section is, for now, a quick note-stash}

As a warm-up, we present a PAD scheme for ECDSA.
\RSW{generalize to EdDSA and prime-field DSA just here, throughout, or just in a comment at the end?}
%
%\paragraph{Notation}
Let $\HH$ be a cyclic group of prime order $q$, generated by $\gHH$.
\RSW{I don't love $\HH$ and I loathe $\gHH$, but maybe it's helpful to use
    different notation here than in the next section. Or maybe this notation
    is so standard it doesn't matter.}

\rsw{maybe standardize on party names, i.e., `challenger' (or `sender'),
    `signer' (or `receiver'), and `verifier'.}

\paragraph{Algorithm $\text{sendTokens}(\PK)$:} To send tokens to public key $\PK = \gHH^x$, do:
\begin{itemize}
\item generate a random $s \rgets \range{q}$
\item output $(c, s)$ where $c \gets \PK\cdot\gHH^s$
\end{itemize}

\rsw{consider moving first optimization note from \S\ref{sec:rsa:airdrop} to
    here (and check for redundancy with sendTokens def in \S\ref{sec:defs}).}

\paragraph{Algorithm $\text{sign}(\SK, (c, s), m)$:}
To withdraw the funds associated with the token $c$ the user signs a
    withdrawal message $m$ using the fact that
$c = \PK\cdot\gHH^s = \gHH^{x+s}$.
%
That is, $c$ is a public key with corresponding secret key $x + s$, and
    $\mathit{sig}$ is an ECDSA signature on $m$ under this key.

\paragraph{Algorithm $\text{verify}(c, m, \mathit{sig}):$}
Accept iff $\mathit{sig}$ is a valid ECDSA signature on $m$ under $c$.


\paragraph{Security}
\begin{itemize}
\item Privacy: if $s$ is uniformly random, then $(c, x+s)$ is a random ECDSA key pair.

\item Unforgeability: an adversary that can forge can break ECDSA (pick $s$ s.t. $c$ is target pubkey)
\end{itemize}
