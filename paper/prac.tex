\section{Practical considerations for private airdrops}

\rsw{stashed note from talk with DB}
Simultaneously using the schemes of \S\ref{sec:dsa} and \S\ref{sec:rsa}
    on the same blockchain allows distinguishing between users based on
    key type.
Can work around this using generic ZKPK, at substantial performance cost.
(Alt narrative: talk about generic ZKPK first, then specialize to get better
    perf. at cost of distinguishing.)

\rsw{stashed note from chat with Henry and Dima}
Possible objection: why doesn't sender just generate a bunch of phony keys
    and send all money to self?
In other words, why is this scheme better, from the verifiers' point of view,
    than the naive approach, i.e., sender generating a fresh key pair and
    sending secret key to recipient?
(Note that while the naive approach does not have existential unforgeability,
    that property does not prevent self-dealing if sender knows secret key.
Another way to say this is that the security property we need for the
    airdrop-to-Github-users case is stronger than existential unforgeability.)

One answer: for each RSA key $\PK = (n, e)$ to be paid, sender publishes
    the challenge $c = g^nh^s$ and (separately, i.e., without revealing
    correspondence) sender also publishes
$(\PK, s')$, $s' = E(\PK, (c',s)))$, $c' = H(g^nh^s)$.
In other words, the sender publishes a list of purported airdrop recipients
    separately from the list of airdrop challenge tokens.

If sender was dishonest with respect to any key, i.e., if for some $(\PK,
    s')$, any of $s', c', c, s$ is invalid, key's owner can reveal $(\hat{c}',
    \hat{s}) = D(\PK, s')$.
Then verifiers check that $s' = E(\PK, (\hat{c}', \hat{s}))$, and that
    either no corresponding $c$ exists, or that $c$ is not a commitment
    to $n$ with opening $s$.

\rsw{NOTE: see notes from DB regarding above (shuffle, etc)}

